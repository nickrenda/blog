\documentclass[12pt]{amsart}
\usepackage{fullpage}
\usepackage[parfill]{parskip}
\usepackage{multirow}

\begin{document}
\pagenumbering{gobble} % remove page numbers
\title{\large{
Stat 101 Sections G and H: Chapter 2 Quiz
}}
\author{
Name: \texttt{\_\_\_\_\_\_\_\_\_\_\_\_\_\_\_\_\_\_} \hfill Section: \texttt{\_\_\_\_\_\_} \\
}
\maketitle

Data collected from \texttt{www.the-numbers.com} contains information on 348 films from 2010 that were given a rating by the Motion Picture Association of America (MPAA) and affiliated with a certain genre. You will only be asked to work with a subset of the data, as shown below, to make calculations easier. We are interested in how film genres differ. Thus, genre is the explanatory variable and rating is the response that we will use to compare genres. Remember that when you asked to give a distribution, use proportions or percents. 

\begin{table}[h]
\caption{Movies}
\begin{tabular}{c l | c  c  c | c}
& & \multicolumn{3}{ | c | }{Response}  \\
& & PG & PG-13 & R & Total \\
\hline
\multirow{3}{*}{Genre} & Action/Adventure & 5 & 5 &  &  \\
& Comedy & 5 &  & 10 & 22 \\
& Drama &  & 4 & 5 &  \\
\hline
& Total &  & 16 & 18 & 46 \\
\end{tabular}
\end{table}

\subsection*{1} There are 6 values missing from the table. Fill in the missing values.

\subsection*{2} Give the marginal distribution of genre. Hint: If you only write down one number, you are wrong. 

\vspace{15mm}

\subsection*{3} Give the marginal distribution of the response (rating).

\vspace{15mm}

\subsection*{4} Give the conditional distribution of rating given that genre is ``Comedy."

\vspace{15mm}

\subsection*{5} Is there an association between the two variables? There is no right or wrong answer as long as you provide a solid argument.

\end{document}

